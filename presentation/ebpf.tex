\documentclass[aspectratio=169]{beamer}

\usepackage[T1]{fontenc}
\usepackage[english]{babel}
\usepackage{dirtytalk}
\usepackage{pgfpages}
\usepackage{listings}
\usepackage{linehighlight}
\usepackage{color}
\usepackage{xparse}
\usepackage{xfp}

\usepackage{blindtext}
\usepackage{lipsum}

\title{Introduction to eBPF}
\author{Martin Ertsås\\ \texttt{martiert@gmail.com}}
\date{\today}

\usetheme{ciscolysaker}
\setbeameroption{show notes on second screen=right}

\definecolor{codehighlight}{rgb}{0.8, 0.95, 0.8}
\definecolor{codebackground}{rgb}{0.9, 0.9, 0.9}


\usepackage{graphicx}

\lstdefinelanguage{diff}{
 morecomment=[f][\color{Green}]{+},
 morecomment=[f][\color{Red}]{-},
}

\lstdefinelanguage{asm}{
  morekeywords={load, jmp, jeq, jge, jgt, ret},
}
\lstdefinestyle{cpp}{
  language=C++,
  morekeywords={
    __u32,
    __u64,
    __s32,
    __s64,
    uint8_t,
    uint16_t,
    uint32_t,
    uint64_t,
    int8_t,
    int16_t,
    int32_t,
    int64_t,
  },
}

\lstset{
 frame=single,
 language=diff,
 morekeywords={uint8_t, ssize_t},
 keywordstyle=\color{Blue},
 showspaces=false,
 showstringspaces=false,
 backgroundcolor=\color{codebackground},
 literate={å}{{\aa}}1
}
\newenvironment{SplitView}[2]
  {
    \begin{columns}
      \begin{column}{0.35\textwidth}
        \centering
        \includegraphics[width=\textwidth, keepaspectratio]{#1}
      \end{column}
      \begin{column}{0.65\textwidth}
        {\Large\color{Blue}{#2}}
        \vskip 1em
  }
  {
      \end{column}
    \end{columns}
  }
\newcommand{\demo}[1]{\begin{frame}{DEMO!}\note{#1}\end{frame}}

\begin{document}

\begin{frame}
  \titlepage

  \note{
    \begin{itemize}
      \item Ask questions whenever
      \item Feel free to interrupt
      \item If I forget breaks, shout out
    \end{itemize}
  }
\end{frame}

\begin{frame}
  \begin{SplitView}{images/mertsas.jpg}{\$ Whoami}
    \centering
    \lstinputlisting[basicstyle=\small]{examples/whoami.txt}
  \end{SplitView}

  \note{
    \begin{itemize}
      \item Works for Cisco Norway
      \item Been working a couple years now with integrating chromium in our embedded devices
      \item Mainly interested in systems programming, security, and training
      \item Likes board games, RPG games, and cycling
    \end{itemize}
  }
\end{frame}

\begin{frame}
  \begin{SplitView}{images/mertsas.jpg}{I have no idea what I'm doing}
    \begin{itemize}
      \item<2-> I wanted to experiment with eBPF
      \item<3-> So I submitted this talk
      \item<4-> This might have been a mistake
      \item<5-> Definitely do not use any code from this presentation
      \item<6-> It will blow up in your face
    \end{itemize}
  \end{SplitView}

  \note{
  }
\end{frame}

\begin{frame}
  \begin{SplitView}{images/mertsas.jpg}{What is eBPF?}
    \begin{itemize}
      \item A way to program the kernel without building a new kernel or writing kernel modules
      \item A programing language, evaluated by the kernel
      \item Event driven programs, running when the kernel passes a pre-defined hook
      \item Can be controlled from userspace
    \end{itemize}
  \end{SplitView}

  \note{
    Write kernel features without:

    \begin{itemize}
      \item Modifying the kernel
      \item Submit the patch upstream
      \item Wait for kernel release
      \item Wait for distro to pick up the new kernel
    \end{itemize}
  }
\end{frame}

\begin{frame}
  \begin{SplitView}{images/mertsas.jpg}{But, isn't running code in the kernel dangerous?}
    \begin{itemize}
      \item Runs through a verifyer on every program load
      \item<2-> Process loading the eBPF program needs correct capabilities
      \item<2-> The program does not crash or harm the system
      \item<2-> The program runs to completion
    \end{itemize}
  \end{SplitView}

  \note{
  }
\end{frame}

\begin{frame}
  \begin{SplitView}{images/mertsas.jpg}{Why would you like to use this?}
    \begin{itemize}
      \item Monitoring and tracing of processes
      \item Network monitoring
      \item Making network decisions
      \item To be one of the cool kids
    \end{itemize}
  \end{SplitView}

  \note{
    Our use case was to put a specific process into a cgroup on creation, to limit its memory usage
  }
\end{frame}

\begin{frame}
  \begin{SplitView}{images/mertsas.jpg}{Abstractions and tools}
    \begin{itemize}
      \item Foo
    \end{itemize}
  \end{SplitView}

  \note{
  }
\end{frame}

\begin{frame}
  \begin{SplitView}{images/mertsas.jpg}{C interface}
    \begin{itemize}
      \item Clang can compile to bpf programs
    \end{itemize}
  \end{SplitView}

  \note{
  }
\end{frame}

\begin{frame}{Simple ebpf program}
  \lstinputlisting[basicstyle=\small]{examples/simple.ebpf.c}

  \note{
    \begin{itemize}
      \item $SEC$ defines an ELF section to use
      \item A GPL compatible license is required
      \item $BPF_CORE_READ$ reads entries from a struct
      \item bpf\_probe\_read\_kernel\_str copies from an unsafe kernel address
    \end{itemize}
  }
\end{frame}

\begin{frame}{Simple loading program}
  \lstinputlisting{examples/simple.host.c}

  \note{
    The simple\_\_ functions are auto generated from the ebp program using bpftool
  }
\end{frame}

\demo{
  \begin{itemize}
    \item bpftool prog to show that it is loaded
    \item cat /sys/kernel/debug/tracing/trace\_pipe
  \end{itemize}
}

\begin{frame}
  \begin{SplitView}{images/mertsas.jpg}{Communicate to userspace}
    \begin{itemize}
      \item .bss section
      \item ebpf maps
    \end{itemize}
  \end{SplitView}
\end{frame}

\begin{frame}
  \lstinputlisting[caption=ebpf]{examples/bss.diff}
  \lstinputlisting[caption=host]{examples/bss.host.diff}

  \note{
    \begin{itemize}
      \item Adding an integer to the .bss section
      \item Using that from the ebpf program
      \item Exposed in the skeleton on the host side
      \item Pretty much one-way
      \item Very limited
    \end{itemize}
  }
\end{frame}

\begin{frame}
  \begin{SplitView}{images/mertsas.jpg}{ebpf maps}
    \begin{itemize}
      \item Multiple map types
    \end{itemize}
  \end{SplitView}
\end{frame}

\begin{frame}
  \begin{SplitView}{images/mertsas.jpg}{Creating a map}
    \lstinputlisting{examples/ringbuffer.define.diff}
  \end{SplitView}
\end{frame}

\begin{frame}{Populating a ringbuffer from ebpf}
  \lstinputlisting{examples/ringbuffer.bpf.diff}
\end{frame}

\begin{frame}{Reading a ringbuffer from ebpf}
  \lstinputlisting{examples/ringbuffer.userspace.diff}
\end{frame}

\begin{frame}{Consuming data}
  \lstinputlisting{examples/ringbuffer.callback.cpp}
\end{frame}

\demo{}

\begin{frame}
  \begin{SplitView}{images/mertsas.jpg}{Communicating from userspace}
    \begin{itemize}
      \item Very similar to communicating to userspace
      \item Can send over more complex types
      \item Can easily keep several entries
    \end{itemize}
  \end{SplitView}
\end{frame}

\begin{frame}
  \begin{SplitView}{images/mertsas.jpg}{Path of commands map}
    \lstinputlisting{examples/command\_name.ebpf.diff}
  \end{SplitView}
\end{frame}

\begin{frame}
  \begin{SplitView}{images/mertsas.jpg}{Using the commands map}
    \lstinputlisting{examples/command\_name.ebpf.check.diff}
  \end{SplitView}
\end{frame}

\begin{frame}
  \begin{SplitView}{images/mertsas.jpg}{Populating from userspace}
    \lstinputlisting{examples/command\_name.userspace.diff}
  \end{SplitView}
\end{frame}

% Internal communication between exec and exit

% Connect a network ebpf program to a cgroup

\begin{frame}
  \begin{SplitView}{images/Wictor.jpg}{Thank you}
  \end{SplitView}
\end{frame}

\end{document}
